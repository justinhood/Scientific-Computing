\documentclass[letterpaper,10pt]{article}
\usepackage[top=2cm, bottom=1.5cm, left=1cm, right=1cm]{geometry}
\usepackage{amsmath, amssymb, amsthm,graphicx}
\usepackage{fancyhdr}
\pagestyle{fancy}

\lhead{\today}
\chead{Scientific Computing Meeting 6}
\rhead{Justin Hood}

\newcommand{\Z}{\mathbb{Z}}
\newcommand{\Q}{\mathbb{Q}}
\newcommand{\R}{\mathbb{R}}
\newcommand{\C}{\mathbb{C}}
\newtheorem{lem}{Lemma}

\begin{document}
\begin{enumerate}
\item Webwork Complete.
\item Consider the system of equations,
\begin{align*}
0.003x_1+59.14x_2 &= 59.17\\
5.291x_1-6.130x_2 &= 46.78
\end{align*}
First, we test that $[10,1]^T$ is a solution,
\[\begin{bmatrix}
0.003 & 59.14\\ 5.291 & -6.130
\end{bmatrix}\begin{bmatrix}
10\\1
\end{bmatrix}=\begin{bmatrix}
0.03+59.14\\52.91-6.130
\end{bmatrix}=\begin{bmatrix}
59.17\\46.78
\end{bmatrix} \]
As desired. We now apply gaussian elimination to the system with 4-digit arithmetic:
\begin{align*}
\left[\begin{array}{cc|c}
0.003 & 59.14 & 59.17\\ 5.291 & -6.130 & 46.78
\end{array}\right] &\to \left[\begin{array}{cc|c}
0.003 & 59.14 & 59.17\\ 0.001 & -104300 & -104400
\end{array}\right] && R_2-1764R_1\\
&\to \left[\begin{array}{cc|c}
0.003 & 59.14 & 59.17\\ 0.000 & -104300 & -104400
\end{array}\right] && R_2-\frac{1}{3}R_1\\
&\to \left[\begin{array}{cc|c}
0.003 & 59.14 & 59.17\\ 0.000 & 1 & 1.001
\end{array}\right] && \frac{-1}{104300}R_2\\
&\to \left[\begin{array}{cc|c}
1 & 19710 & 19720\\ 0.000 & 1 & 1.001
\end{array}\right] && \frac{1}{0.003}R_1\\
&\to \left[\begin{array}{cc|c}
1.000 & 0.000 & -10\\ 0.000 & 1 & 1.001
\end{array}\right] && R_1-19710R_2\\
\end{align*}
We see that the result is then $[-10, 1.001]^T$. The four digit arithmetic without pivots leads to dramatically incorrect results.
\item Now, we apply the partial pivot method,
\begin{align*}
\left[\begin{array}{cc|c}
0.003 & 59.14 & 59.17\\ 5.291 & -6.130 & 46.78
\end{array}\right] &\to \left[\begin{array}{cc|c}
5.291 & -6.130 & 46.78\\0.003 & 59.14 & 59.17
\end{array}\right] && \text{Move largest value to R1}\\
&\to \left[\begin{array}{cc|c}
5.291 & -6.130 & 46.78\\0.000 & 59.14 & 59.14
\end{array}\right] && R_2-0.006R_1\\
&\to \left[\begin{array}{cc|c}
5.291 & -6.130 & 46.78\\0.000 & 1.000 & 1.00
\end{array}\right] && \frac{1}{59.14}R_2\\
&\to \left[\begin{array}{cc|c}
5.291 & 0.000 & 52.91\\0.000 & 1.000 & 1.000
\end{array}\right] && R_1+6.130R_2\\
&\to \left[\begin{array}{cc|c}
1.000 & 0.000 & 10\\0.000 & 1.000 & 1.000
\end{array}\right] && \frac{1}{5.291}R_1
\end{align*}
We see that the partial pivoting method retains more information for solving. This method is more numerically stable than simple elimination.
\item See Jupyter Notebook
\item See Jupyter Notebook
\item See Jupyter Notebook
\end{enumerate}
\end{document}
